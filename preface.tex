\newthought{The Iroquois Indians}, during the 14th through 17th century, 
tortured, killed, and ate prisoners of war. \cite{abler1980} This was motivated, at least in part, by the belief that you could absorb your enemy's power by consuming his flesh. If you ate the heart of a courageous opponent, you would gain his courage.

Reading a book is the modern equivalent of eating the heart of the author, with the modest benefit of actually working. By reading a book, you are imbued with some of the powers of the author, a portion of his knowledge and habits of thought. Your power grows with each book you consume, and this book is no exception. 

This is a book about heart-eating. It's about how you ought to go about absorbing knowledge, how to do it quickly and effectively. It's about learning difficult things, from representation theory to
Mandarin to rocket science. It's about exploiting your own mind for fun and profit. It's about internalizing knowledge, coming to know it in your bones, owning it. It's about moving from a shallow understanding to a deep understanding. To paraphrase Grothendieck,\marginnote{``To have broken these bounds they would have to rediscover in themselves that capability which was their birthright, as it was mine: The capacity to be alone.''}\cite{grothendieck1985recoltes} this is a book about rediscovering that capability which is your birthright: the capacity to learn. It's about the pleasure of finding things out and the kick in the discovery. 

\newthought{The core of this book} is divided into two components: theory and the fruit of that theory. The theory provides answers to questions like, ``How do humans learn?'' or ``How do humans process and store information?'' The fruit of the theory is focused on answers to practical questions like, ``What can I do to learn more effectively?'' The emphasis of this part of the text is on applications.

Theory provides a framework for understanding how the world fits together. It's a skeleton on which you can hang new information and use to make predictions in novel situations. It'll enable you to explain why something works and to design interventions in a way such that they ought to work. It allows for a certain amount of flexibility. If you know why the principles that underly a particular technique work, you know which components are important and which aren't and you can identify which you can change.

But, beyond all eslse, we are interested in useful theories; we want something that will empower us and move us closer to our goals. The triumphs of the physical sciences are so impressive because of their applications: motor engines, skyscrapers, the Hoover dam, the atomic bomb, and the moon landing. We're after the moon landing equivalent of the mind: mental gymnastics that were once restricted to science fiction.

This book will enable you to learn things that once seemed impossible. With enough effort, you can learn to multiply large numbers mentally, memorize shuffled decks of cards in a few minutes, or play blindfold chess against several people at once. 

In pursuit of this goal, I've reviewed broad swathes of the available literature, drawing from a variety of disciplines, from anthropology to educational psychology to brain science. I've ranged still further when seeking for metaphors, adopting a good one wherever it's to be found. The mind can be a clock or a sponge or a network. It can be like a muscle or an elephant. Learning can be a dance or a war or even like eating the heart of a sworn enemy. Richard Feynman was a master of the metaphor and I've strived to capture some of that Feynman-esque feel within these pages.

There's more than one way to nail down a concept, and I've exploited them all. Through pictures, words, metaphors, and above all the concrete example. The explanations are meant to be accessible without being boring. In the pursuit of clarity, I've emphasized plain language and the elimination of jargon. I have a fondness for concrete examples and as a result the text is littered with them. They serve to drive a point home.

\newthought{I have assumed nothing} of the reader beyond a readiness to engage the intellect and a certain amount of curiosity about the world. You must also a possess willingness to try new things. If you are going to go about learning the same way that you always have once you've read this book, ``business as usual,'' then there's no reason to continue any further. You can just keep doing what you've been doing and get the same results.

Emerson perhaps put it best when he wrote, ``All life is an experiment. The more experiments you make, the better.''\cite{emerson1984emerson} All improvements to the status quo are changes. The more you try, the more opportunities you have to discover something truly remarkable, something life-changing. If you refuse to try new food, you will never discover a dish better than anything you've tried before. It's the same with learning techniques. If you've only ever read through a textbook while taking notes and don't try anything else, you'll never discover a more effective way of doing things.

\newthought{You will notice} that the margins have thus far contained citations
and quotations. The intention is to give credit where it is due, \marginnote{``If I
have not seen as far as others, it is because giants were standing on my
shoulders.'' -- Hal Abelson} although I will
no doubt fall short of such a standard, and to make otherwise obscure references less
so. Where I could quote a Fields Medalist or Nobel Laureaute to support a point, I
have done so. There's something more definitive about hearing it from one of
the greats and it's often been the case that someone has expressed a point with more
precision than I could hope to.

The information in the margins serves, in general, to illuminate the body of the
text without distracting from the main points. In addition to quotes and
sources, it contains comedic relief, tangential points, and miscellaneous
trivia. A book is a journey through the mind of the author and I've used (or
abused) the margins as a route through dusty passages that otherwise would not
see the light of day. One wants not just ideas, but context, connections, and
whimsy along with those ideas, and this has been packed into the sidebar. 
