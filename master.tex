\documentclass{tufte-book}
\usepackage{mathtools}  % loads »amsmath«
\usepackage{physics}
\usepackage{amssymb} %Some extra symbols
\usepackage{makeidx} %If you want to generate an index, automatically
\usepackage{graphicx} %If you want to include postscript graphics
\usepackage{attrib}
\usepackage{graphicx}
\usepackage{microtype}
\usepackage{hyperref}
\usepackage[utf8]{inputenc}

\hypersetup{colorlinks=true}
%\usepackage{mystyle} %Create your own file, mystyle.sty where you put all your own \newcommand statements, for example.

\newcommand{\openepigraph}[2]{ % This block sets up a command for printing an epigraph with 2 arguments - the quote and the author
\begin{fullwidth}
\sffamily\large
\begin{doublespace}
\noindent\allcaps{#1}\\ % The quote
\noindent\allcaps{#2} % The author
\end{doublespace}
\end{fullwidth}
}

\author{Robert Seaton}
\title{Managing Modernity}

\begin{document}
\maketitle
\thispagestyle{empty}
\openepigraph{If knowledge can create problems, it is not through ignorance we can solve them.}{\itshape Isaac Asimov}
\vfill
\openepigraph{Use only that which works, and take it from any place you can find it.}{\itshape Bruce Lee}
\vfill
\openepigraph{Most haystacks do not have even a needle.}{}
\frontmatter
\tableofcontents

\mainmatter
\begin{quote}
At the rate of progress since 1800, every American who lived into the year 2000
would know how to control unlimited power. He would think in complexities
unimaginable to an earlier mind. He would deal with problems altogether beyond
the range of earlier society. To him the nineteenth century would stand on the
same plane with the fourth -- equally childlike -- and he would only wonder how
both of them, knowing so little, and so weak in force, should have done so much.
\attrib{Henry Adams, \textit{The Education of Henry Adams}}
\end{quote}

\begin{quote}
It is change, continuing change, inevitable change, that is the dominant factor
in society today. No sensible decision can be made any longer without taking
into account not only the world as it is, but the world as it will be\ldots
\attrib{Isaac Asimov, \textit{Asimov on Science Fiction}}
\end{quote} 

\newthought{In 1907, Henry Adams published} his autobiography, \textit{The Education of Henry
Adams}. In it, he simultanesouly laments and marvels at the technological change
he has witnessed. When he writes about the impossibility of
understanding it all, his despair is palpable. 

Today, we can only laugh. Too much change? In \textit{1907}? Sure, man taming electricty may have
been a bit of a shock (pun intended), but it was 46 years until it had reached a
quarter of Americans -- more than enough time for the novelty of the thing to
wear off. Childlike, indeed.

Compare that to the inventions since then: It took the telephone 35 years to teach a quarter of Americans, radio
took 31 years, 26 years for television, 16 for the personal computer, 13 for
mobile phones, and a mere 7 for the world wide web.\bigskip

\includegraphics[width=\textwidth]{graphics/accelerating-change}
\bigskip

If we accept, then, that the dissemination of the world wide web and electrity are
roughly comparable, well, some arithmetic suggests that the world is
changing about 6.5x faster today than it was when Henry Adams published his
autobiography. Of course, I'm sure you've felt it -- the constant barrage of the
``next big thing'' that you're tasked with mastering: for web programmers, the
latest and greatest JavaScript framework. For teens, Vine and Snapchat. For
workers generally, your employer's latest iteration of new time tracking
software. 

And this rate of change is accelerating. We're expected to adopt each technology faster than the last.

\newthought{This trend isn't limited to technology.} It's more than
advances like television and radio. In the 13th century, Roger Bacon argued that
it was impossible to master mathematics in less than 30 to 40 years. Today,
roughly equivalent material is routinely taught to high school students.

Sports, too, are no exception. If we transported Olympic swimmers from 1896 to now, they would not even qualify. \cite{ericsson2006influence}

\newthought{Further, consider the sheer amount} of information being produced. About a million books are published each year. That's more books each year than were published
by Western civilization in the sixth, seventh, eighth, ninth, tenth, and
eleventh centuries combined -- a 600 year span. More books were published \textit{just
today} than France did during a 100 year span from the years 600 AD to 700
AD.\cite{buringh2009charting} If that wasn't convincing enough, Google's
executive chairman, Eric Schmidt, estimates that we create more information
every two days than was created from the dawn of civilization until 2003.

There is, then, this expectation that students and workers today will absorb
more and more information in less and less time, and we have no reason to expect
this trend to abate. In fact, it looks as if it will only get worse.

\section{What to expect}

\newthought{Don't panic.} The entire point of this book is to make the world more manageable -- to teach you a few techniques and principles that will supercharge your ability to absorb and retain information. 

In pursuit of this goal, the book is organized into three parts:

\begin{enumerate}
  \item Finding, filtering, and managing external information
  \item Understanding
  \item Retaining knowledge
\end{enumerate}

\chapter{Finding, Filtering, and Managing Information}
\begin{quote}
  In an information-rich world, the wealth of information means a dearth of something else: a scarcity of whatever it is that information consumes. What information consumes is rather obvious: it consumes the attention of its recipients. Hence a wealth of information creates a poverty of attention and a need to allocate that attention efficiently among the overabundance of information sources that might consume it.
\attrib{Herbert Simon\cite{simon1971designing}}
\end{quote}

Stuff to include:

- Textbooks are the best way to get up to date on a subject.
- OEIS
- Human expertise is still immensely beneficial in today's world.
- Annotated bibliographies are worth their weight in gold.
- the important of knowing the name of a keyword, example: externalities. quote: know the name of a spirit, have control over it.
- A few tips on effective googling, keywords.
- Google scholar.
- A note on the value of information. You can sell it. (Maybe an intro to VoI caluclations, too?)

\chapter{Understanding}

\begin{quote}
It isn't that they can't see the solution. It's that they can't see the problem.
\attrib{Gilbert K. Chesterton}
\end{quote}

- cognition is relative; metaphor

\section{Perspective Collection}

\begin{quote}
A good stack of examples, as large as possible, is indispensable for a thorough
understanding of any concept, and when I want to learn something new, I make it
my first job to build one.
\attrib{Paul Halmos}
\end{quote}

I know a bit of abstract algebra. And I know how to cook. It turns out, cooking
has gotten me a lot more attention from women than cooking ever has -- who knew,
right? (Don't answer that.)

You know, the two subjects have a lot in common. And I don't mean this in some
deep, ``the reality of the universe is contained in mathematics and mathematics
is contained in the reality of the universe'' babble. Maybe symmetry in recipes
makes them taste better. I don't know.

You'll need a more informed mathematician to speak about that.

What I mean is that learning either of these subjects can teach you the same
lesson about human cognition. That lesson? \textbf{Collect multiple
  perspectives.}

When I want to cook something, I never read just one recipe -- I read five or
ten. With five or ten recipes, you can pattern match. Ask yourself, ``What's
constant across all of these recipes?'' And once you know the answer to that,
well, you've sort of just figured out the heart and soul of a dish.

Wittgenstein can argue all he wants about the definition of a game but, read 5
recipes for the same thing, and you'll have a definition of that dish. And you
don't even need some platonic cookbook sent down from on high. All you need is
an internet connection.

Once you understand the core of a dish, you're free to improvise. And I think
that's really what cooking is about. Grasping the underlying pieces of a recipe
that make it \textit{it} and then adding your twist to it. A delicious cinnamon
twist, maybe.

Abstract algebra works in the same manner. If you want to know what a field,
group, or a monoid is, you can't just read one recipe, one treatment. You need
to read five or ten. Flip through a dozen textbooks. Once you've deciphered those, you ought to have a good
idea about the object in question.

Or take mathematical proof, for instance. What's the point of knowing multiple
proofs for the same theorem? Why, once something has been proven, do
mathematicians continue searching for cleaner, more elegant proofs? Or proofs by
different means? Maybe more elementary ones.

Why bother? It's been proven. It's valid. Who cares?

The purpose of multiple proof, and even mathematical proof in general, is not to
erect the truth of something in such a way that it's impervious to sane
criticism. That's just a side effect.

The real point of a proof is to illuminate the mathematics -- we're after
insight, not correctness guarantees. And that's the point of multiple proofs:
each, if we're lucky, sheds some additional light on the structure. Our mental
picture becomes sharper, more defined.

We start to feel that we know.

So we've generalized this from recipes, to abstract algebra, to math and proofs
as a whole, but don't stop there: \textbf{to understand anything, anything at all, in
  any significant way, collect perspectives on that thing}.

5, 7, 13. The number of perspectives you need varies. With something simple,
maybe none are needed. With something complex, maybe you need to read 100 takes
on the same idea.

In fact, this idea is so powerful that it's not limited to humans. Hell, it's
not even limited to organic life. There have recently been some advances in
automated theorem proving by training programs on several different proofs of
the same theorem -- exactly the idea I've just told you about.

% tk find link/citation

\section{Debugging Confusion}
\begin{quote}
A person taking a test or playing a piece of music is executing a program, a
deterministic procedure.  If your program has a bug, then you'll get a whole
class of problems wrong, consistently.
\attrib{celandine13, \href{http://celandine13.livejournal.com/33599.html}{the most insightful LiveJournal entry of all time}}
\end{quote}

Lend me your imagination for a minute.

One day, you're cleaning your attic. Light filters in through some crooked
blinds. It highlights the (significant) dust in the air, and you can make out
each speck, like a tiny alien race, come to colonize your attic. 

It's hot. You wonder why you volunteered for this project in the first place.

You're emptying a chest of drawers and you stumble across... well, it's what
looks like a glass sphere -- sorta a snow globe, but sans base.

It has to be old, given the amount of grime that it's accumulated. The muck
coats your fingers. Your nose curls and you grimace in a moment of disgust, and then you grab a
towel -- first each finger, and then you polish off the mutant snow
globe.

And, as it starts to gleam, your attention latches onto the tiny glass planet,
until, with a start, you realize that you can't feel your body. It's like your
consciousness has been possessed, and you can't steal your eyes away from the
ball.

Bit by bit, the attic around you disappears like television static, and you find your awareness floating,
bodyless, over a boy hunched over a math problem set and a sheet of paper. He
grips a pencil and, by his side, there lies the nub of what used to be a meaty
eraser.

As he flips through the back of the calculus book, the look of concentration on
his face is replaced with one of expectation. He lands on a page. 

His fist strikes the table.

There's a frustrated bellow-grunt, and you hear him say, "I'm too stupid. I'll
never get this."

You, too, are swept away in a moment of shared frustration, as your mind is
doubly hijacked: first by the snow globe, and now by empathy.

You wonder why the ball has sent you here, to this particular time and place,
and then...

\section{What's in a skill? }

Here's something I dislike: when people describe something as algorithmic as an
insult. Oh, \textit{that}, it's so algorithmic. A computer could do that.

Like when colleges tout that oh, well, we have humans review each child's
application, unlike some other places, who just let a \textit{computer} do
it.

Except, when you look at the empirical data, it's very, very tough to find a
situation where a human's predictive ability beats out a computer. 

This is well-documented in Robyn Dawes's landmark paper, "The robust beauty of
improper linear models in decision making." The gist of the paper? Basically,
even ghetto,
if-they-were-FDA-regulated-only-Chinese-citizens-would-be-allowed-to-buy-them
models outperform human.

\begin{quote}
This article presents evidence that even such improper linear models are
superior to clinical intuition when predicting a numerical criterion from
numerical predictors. 
\end{quote}

This paper could be considered the precursor to Tetlock's magnum opus, \textit{Expert
Political Judgment}, which found that quantitative models outperform, you know,
everyone.

This is illustrated in the book's most famous graphic, reproduced below, where
statistical models wrek the human and chimpanzee competiton.

\includegraphics{expert-political-judgment}

But, right, I'm getting a little carried away here. The point I'm trying to make
is that \textit{algorithms rule}. And, like I've argued before, everything can be
described as an algorithm. The entire scientific enterprise can be described as
algorithm-hunting.

The question, "How does the mind work?" is really looking for a step-by-step
process that the mind goes through to accomplish something -- when you can write
something down in such a way that a program can do, that's when you've
understood it.

If every process can be conceptualized as an algorithm, then, skills must fall
under that umbrella -- a skill is an algorithm, or as the quote at the intro
puts it, "A person taking a test or playing a piece of music is executing a
program, a deterministic procedure."

What are errors, then? Nothing but bugs in your program.

\section{Mistakes are valuable}

Humans are handicapped by our binary notion of correctness -- the notion that an
answer on a test is either correct or not. That there are no degrees.

But, after a bit of reflection, it becomes clear that one answer can be more
correct than another, while both can still be technically wrong.

Isaac Asimov has this to say on the subject in "The Relativity of Wrong,"

\begin{quote}
[W]hen people thought the earth was flat, they were wrong. When people thought
the earth was spherical, they were wrong. But if you think that thinking the
earth is spherical is just as wrong as thinking the earth is flat, then your
view is wronger than both of them put together.
\end{quote}
  
Once we abandon the binary notion of right and wrong, the actual nature of
wrongess becomes a bit more clear -- some of the haze dissipates. You will, if
you're paying attention, notice that wrongness is not one thing.

That there are many different ways to be wrong, and each of them have
indentifiable causes and the cause, once removed, eliminates the
wrongness. Thus, the label "stupid" or "bad at" are not so much explanations, as
they are frustrated noises that, when decoded, translate to "you're getting it
wrong but I don't understand the reason why."

When I was a student, I didn't understand this about wrongness and I didn't grok
that \textit{mistakes are evidence.}

Imagine you're a teacher, and you're grading 35 tests on, say, division,
and every test comes back the same: every student has answered that 100 divided
by 10 equals 90.

Are you going to say to yourself, "Well, I guess I just got a really dumb batch
of children this year. That's why they're getting it wrong."

No, of course not. You're going to go back over division with the children, with
an emphasis on the differences between division and subtraction -- because
apparently all the kids think you're asking them to subtract!

When you're learning something on your own, you have to be that teacher. If you
get something wrong, you need to stop, and pay attention to your mistake. With
mathematics, assuming you've kept reasonable notes, you can walk through your
answer, step by step, and see at which point you went wrong.

You can zero in on the bug and destroy it.

That's the process of debugging confusion.

\begin{itemize}
\item First, you get a problem wrong.
\item Then you must infer the cause of the mistake by collecting evidence.
\item Once you have the cause of the mistake, dispel the confusion, crush the bug,
  and update your programming.
\item Repeat this process until you've dispelled all of your bugs and reach the
  correct answer.
\end{itemize}

\section{Deeper Processing}

\section{Read with the intention of teaching}
There is compelling evidence that when students read with the intention of learning the material as well as they can, they learn less than students instructed to learn the material so that they can teach it to someone else (Bargh & Schul, 1980).

%\include{chaptr2}
%\include{chaptr3}
%\include{chaptr4}
%\include{chaptr5}
% Possible chapter: keeping up with new information, findings.
\backmatter
% \include{glossary}
% \include{notat}
\bibliography{references} %The files containing all the articles and books you ever referenced.
\bibliographystyle{plainnat} %The style you want to use for references.

\printindex %Make an index AUTOMATICALLY

\end{document}
