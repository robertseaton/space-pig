\chapter{Understanding}

\begin{quote}
It isn't that they can't see the solution. It's that they can't see the problem.
\attrib{Gilbert K. Chesterton}
\end{quote}

- debugging confusion
- cognition is relative; metaphor

\section{Perspective Collection}

I know a bit of abstract algebra. And I know how to cook. It turns out, cooking
has gotten me a lot more attention from women than cooking ever has -- who knew,
right? (Don't answer that.)

You know, the two subjects have a lot in common. And I don't mean this in some
deep, ``the reality of the universe is contained in mathematics and mathematics
is contained in the reality of the universe'' babble. Maybe symmetry in recipes
makes them taste better. I don't know.

You'll need a more informed mathematician to speak about that.

What I mean is that learning either of these subjects can teach you the same
lesson about human cognition. That lesson? \textbf{Collect multiple
  perspectives.}

When I want to cook something, I never read just one recipe -- I read five or
ten. With five or ten recipes, you can pattern match. Ask yourself, ``What's
constant across all of these recipes?'' And once you know the answer to that,
well, you've sort of just figured out the heart and soul of a dish.

Wittgenstein can argue all he wants about the definition of a game but, read 5
recipes for the same thing, and you'll have a definition of that dish. And you
don't even need some platonic cookbook sent down from on high. All you need is
an internet connection.

Once you understand the core of a dish, you're free to improvise. And I think
that's really what cooking is about. Grasping the underlying pieces of a recipe
that make it \textit{it} and then adding your twist to it. A delicious cinnamon
twist, maybe.

Abstract algebra works in the same manner. If you want to know what a field,
group, or a monoid is, you can't just read one recipe, one treatment. You need
to read five or ten. Flip through a dozen textbooks. Once you've deciphered those, you ought to have a good
idea about the object in question.

Or take mathematical proof, for instance. What's the point of knowing multiple
proofs for the same theorem? Why, once something has been proven, do
mathematicians continue searching for cleaner, more elegant proofs? Or proofs by
different means? Maybe more elementary ones.

Why bother? It's been proven. It's valid. Who cares?

The purpose of multiple proof, and even mathematical proof in general, is not to
erect the truth of something in such a way that it's impervious to sane
criticism. That's just a side effect.

The real point of a proof is to illuminate the mathematics -- we're after
insight, not correctness guarantees. And that's the point of multiple proofs:
each, if we're lucky, sheds some additional light on the structure. Our mental
picture becomes sharper, more defined.

We start to feel that we know.

So we've generalized this from recipes, to abstract algebra, to math and proofs
as a whole, but don't stop there: \textbf{to understand anything, anything at all, in
  any significant way, collect perspectives on that thing}.

5, 7, 13. The number of perspectives you need varies. With something simple,
maybe none are needed. With something complex, maybe you need to read 100 takes
on the same idea.

In fact, this idea is so powerful that it's not limited to human, or even
organic life. 
