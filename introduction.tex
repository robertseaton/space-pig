\begin{quote}
At the rate of progress since 1800, every American who lived into the year 2000
would know how to control unlimited power. He would think in complexities
unimaginable to an earlier mind. He would deal with problems altogether beyond
the range of earlier society. To him the nineteenth century would stand on the
same plane with the fourth -- equally childlike -- and he would only wonder how
both of them, knowing so little, and so weak in force, should have done so much.
\attrib{Henry Adams, \textit{The Education of Henry Adams}}
\end{quote}

\begin{quote}
It is change, continuing change, inevitable change, that is the dominant factor
in society today. No sensible decision can be made any longer without taking
into account not only the world as it is, but the world as it will be\ldots
\attrib{Isaac Asimov, \textit{Asimov on Science Fiction}}
\end{quote} 

\newthought{In 1907, Henry Adams published} his autobiography, \textit{The Education of Henry
Adams}. In it, he simultanesouly laments and marvels at the technological change
he has witnessed. When he writes about the impossibility of
understanding it all, his despair is palpable. 

Today, we can only laugh. Too much change? In \textit{1907}? Sure, man taming electricty may have
been a bit of a shock (pun intended), but it was 46 years until it had reached a
quarter of Americans -- more than enough time for the novelty of the thing to
wear off. Childlike, indeed.

Compare that to the inventions since then: It took the telephone 35 years to teach a quarter of Americans, radio
took 31 years, 26 years for television, 16 for the personal computer, 13 for
mobile phones, and a mere 7 for the world wide web.\bigskip

\includegraphics[width=\textwidth]{graphics/accelerating-change}
\bigskip

If we accept, then, that the dissemination of the world wide web and electrity are
roughly comparable, well, some arithmetic suggests that the world is
changing about 6.5x faster today than it was when Henry Adams published his
autobiography. Of course, I'm sure you've felt it -- the constant barrage of the
``next big thing'' that you're tasked with mastering: for web programmers, the
latest and greatest JavaScript framework. For teens, Vine and Snapchat. For
workers generally, your employer's latest iteration of new time tracking
software. 

And this rate of change is accelerating. We're expected to adopt each technology faster than the last.

\newthought{This trend isn't limited to technology.} It's more than
advances like television and radio. In the 13th century, Roger Bacon argued that
it was impossible to master mathematics in less than 30 to 40 years. Today,
roughly equivalent material is routinely taught to high school students.

Sports, too, are no exception. If we transported Olympic swimmers from 1896 to now, they would not even qualify. \cite{ericsson2006influence}

\newthought{Further, consider the sheer amount} of information being produced. About a million books are published each year. That's more books each year than were published
by Western civilization in the sixth, seventh, eighth, ninth, tenth, and
eleventh centuries combined -- a 600 year span. More books were published \textit{just
today} than France did during a 100 year span from the years 600 AD to 700
AD.\cite{buringh2009charting} If that wasn't convincing enough, Google's
executive chairman, Eric Schmidt, estimates that we create more information
every two days than was created from the dawn of civilization until 2003.

There is, then, this expectation that students and workers today will absorb
more and more information in less and less time, and we have no reason to expect
this trend to abate. In fact, it looks as if it will only get worse.

\section{What to expect}

\newthought{Don't panic.} The entire point of this book is to make the world more manageable -- to teach you a few techniques and principles that will supercharge your ability to absorb and retain information. 

In pursuit of this goal, the book is organized into three parts:

\begin{enumerate}
  \item Finding, filtering, and managing external information
  \item Understanding
  \item Retaining knowledge
\end{enumerate}
