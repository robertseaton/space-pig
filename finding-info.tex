\chapter{Finding, Filtering, and Managing Information}
\begin{quote}
  In an information-rich world, the wealth of information means a dearth of something else: a scarcity of whatever it is that information consumes. What information consumes is rather obvious: it consumes the attention of its recipients. Hence a wealth of information creates a poverty of attention and a need to allocate that attention efficiently among the overabundance of information sources that might consume it.
\attrib{Herbert Simon\cite{simon1971designing}}
\end{quote}

\newthought{In a classroom evironment,} at almost every level, students are given the sources that they're expected to gather information from. They're assigned a textbook or papers to read. Students are never in a position where they're forced to evaluate the quality of several textbooks before deciding which to sink time into.

How to do research, then, is a skill that's rarely taught, and almost never taught well and, of course, that's where this text comes in. There are, broadly speaking, four steps to information gathering.

\begin{enumerate}
  \item Realizing that information would improve of illuminate the situation
  \item Gathering all possibly relevant information and sources
  \item Filtering out the irrelevant bits
  \item Processing the information and recording what's relevant
\end{enumerate}

\section{Recognizing an opportunity}

\newthought{Today, the world is awash} in information, a constant deluge of the stuff. Supposing that tomorrow you were transported to some platonic library, where you never needed to sleep or eat, such that you could read constantly, it would still take about 62,000 years.

What this really means, then, is that for any real life problem, there's almost certainly some kind of information that you should be taking advantage of.

Indeed, it's a common human failure mode to speculate about things that you don't need to speculate about. There was some recently published research, for instance, that indicates that students who spend more time with professors are more moderate in their political beliefs, on average, than those who spend a lot of time with other students.

Predicitably, this led to a flamewar in the comments, centered around whether or not the United States political left has a disproportionate amount of influence on college campuses. One commenter mentioned that, hey, there also exist conservative colleges, so it must all balance out.

This led to a great deal of discussion and theorizing, none of which was evidence based. It took me roughly 15 seconds to Google the first query that I could think of, which reveals that ``72 percent of those teaching at American universities and colleges are liberal and 15 percent are conservative.''

So, no, it probably doesn't balance out.

\newthought{There was nothing exceptional about this discussion.} 

Stuff to include:

- Textbooks are the best way to get up to date on a subject.
- OEIS
- Human expertise is still immensely beneficial in today's world.
- Annotated bibliographies are worth their weight in gold.
- the important of knowing the name of a keyword, example: externalities. quote: know the name of a spirit, have control over it.
- A few tips on effective googling, keywords.
- Google scholar.
- A note on the value of information. You can sell it. (Maybe an intro to VoI caluclations, too?)
- On the effectiveness of filtering. 
- Skimming, table of contents.
