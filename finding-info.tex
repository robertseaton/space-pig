\chapter{Finding, Filtering, and Managing Information}
\begin{quote}
  In an information-rich world, the wealth of information means a dearth of something else: a scarcity of whatever it is that information consumes. What information consumes is rather obvious: it consumes the attention of its recipients. Hence a wealth of information creates a poverty of attention and a need to allocate that attention efficiently among the overabundance of information sources that might consume it.
\attrib{Herbert Simon \cite{simon1971designing}}
\end{quote}

\newthought{In a classroom evironment,} at almost every level, students are given the sources that they're expected to gather information from. They're assigned a textbook or papers to read. Students are never in a position where they're forced to evaluate the quality of several textbooks before deciding which to sink time into.

It's maybe no suprise, then, that most people develop, instead of hearty brain matter, a brothy-soup in their heads that comes \textit{this close} to spilling out whenever they sneeze.

How to do research is a skill that's rarely taught, and almost never taught well. That's where this text comes in. There are, broadly speaking, four steps to information gathering.

\begin{enumerate}
  \item Realizing that information would improve of illuminate the situation
  \item Gathering all possibly relevant information and sources
  \item Filtering out the irrelevant bits
  \item Processing the information and recording what's relevant
\end{enumerate}

\section{Recognizing an opportunity}

\newthought{Today, the world is awash} in information, a constant deluge of the stuff. Supposing that tomorrow you were transported to some platonic library, where you never needed to sleep or eat, such that you could read constantly, it would still take you about 62,000 years to read all the books that currently exist. \marginnote{``Basically what Shannon said is that we define information as the opposite of uncertainty. If you have uncertainty, you don't have information. And once you receive information, uncertainty is being resolved.'' Martin Hilbert}

If that is not tantalizing enough, consider the case of Shaun Winterton. One May day in 2011, he discovered a new species of green lacewing, a dragonfly-esque insect.

Where did he find it, you might wonder -- hiking in the Amazon, maybe? But no. He discovered it from the comfort of his own home while browsing the photo-sharing website Flickr. A new species! Advancing human knowledge! From his couch.\cite{winterton2012charismatic}\footnote{The story continues. Winterton contacted the photographer, Guek Hock Ping, and asked him if he had a specimen. He didn't, but eventually caught one and sent it to Winterton. Winterton then named the species after his own daughter, not Guek.} 

So, yeah, there is a lot of information out there. If, for some real-life problem, you can't think of any already existing information to take advantage of, you're not thinking hard enough.

I've written, for instance, a popular article on the science of eye contact. It opens with an anecdote about 17th century Italian women: in an attempt to enhance their appearance, they would use the essence of the Belladonna plant to dilate their pupils. The only problem? Belladonna, sometimes called nightshade, is poisonous.

Now, I think this is a great fact, the sort of spice that sets an article apart from more common fare. How'd I find it? I did a Google search for ``site:scientificamerican.com eye contact''. The addition of the ``site:'' operator restricts my results to article from that site. I then read the four or five relevant articles that Google returned. One of them included that tidbit -- I don't know where they got it. Perhaps doing the same thing.

Another example? You know that bit about Shaun Winterton discovering the green
lacewing on Flickr? Yeah, I found that with a Google search for
``site:scientificamerican.com finding information.''\footnote{Why
  \textit{Scientific American}? No particular reason beyond the wonderful
  tendency of science writers to pack their articles with just the sort of
  anecdote I'm after.}

\newthought{When people think} of research, a few representative examples spring
to mind: a man in a white coat, a laboratory, a library, heavy books, that sort
of thing.

It's so much broader than that! If you're looking for a long-term relationship
and, as part of that process, you take someone out on a date: \textit{that's
  research}. When you ask a friend how he feels about his new job, more
research! Deconstructing your competitor's last successful ad campaign? That's
research.\footnote{Writing this book came about as the result of research. I
  noticed that so many successful blogs were running email campaigns and
  offering an incentive for signing up and, well, now we're here.}

Research is trying on clothes until you find something that fits your shoulders,
instead of draping across them like a dead raccoon. Research is paying attention
to the words that your customers use in their emails so that, next time you need
content for your company blog, you aren't stuck wondering, ``What do I title
this?'' Research is putting sardines on a bagel or peanut butter in your
ramen\footnote{This is actually pretty good.}, experimenting with how long you
can lie in the sun before getting burned, or cracking your knuckles on only one
hand twice a day for 50 years to see if that hand gets arthritis.\footnote{It
  doesn't. Donald Unger actually carried this out and, for his work, receive the
  Ig Nobel Prize.}

\section{Competitor analysis}

\newthought{Out of all} these examples, there's one that I'd like to emphasize
(literally!): \textbf{competitor analysis}.

Consider the most badass of all the seaslugs: the nudibranch. The name comes
from the latin \textit{nudus}, meaning naked, as it has no shell.

The remarkable thing about the nuibranch is not its nudity, although ``naked sea
slugs'' would make a good name for a band. No, what's cool about the nudibranch
is its powers of absorbtion.

When a nudribranch consumes a plant, for instance, its digestive system actually
absorbs the plant's chloroplasts. This allows the slug to undergo
photosyntehsis. A slug! Photosynthesis!

Similarly, when the slug consumes a jellyfish, it absorbs its stinger cells and
incorporates those into its own body.

Competitor analysis is just like this slug. It's about finding some resource, or
business, or anything you admire, and then picking it apart until you understand
why it works. At that point, you can put together something even better.

My writing has evolved, for example, through the study of those that I
admired. I then started to ape the structure that they used in their own writing
until it became second nature. I started notice patterns, different formats that
people would cling to, and now I often use those.

Or consider the sort of haphazard layout of the typical antique store. It feels
as if you're walking into someone's home, the home of a hoarder. Why are the
stores laid out this way?

Well, I'll bet that the average antique shop owner has no idea \textit{why}. He
probably just thought, ``Well, this is what Thomas Goode's shop looks like in
London and, if it's good enough for him, it's good enough for me.''

But it turns out that there's good reason for this: the clutter evokes a
psychological response in customers. They feel like they're stumbling on buried
treasure (perhaps literally) and, as a result, are more likely to buy.

As another example, email subscription forms are a subject near to my
heart. When writing my own, I could have started from scratch and then split
tested each iteration, figuring out what works best.

But why bother with that when large campaigns, like Ramit Sethi's \textit{I Will
  Teach You to be Rich} have already done this? You can bet that he's spent a
couple hundred hours tweaking his own copy, and there's no need to duplicate
that effort. You can go on his website, see what works, and then riff on that
for writing your own.

\textbf{This technique applies to everything. When you see something impressive,
  figure out what makes it tick.} Their success can be your success.

\section{Wait, but who's the competition?}

As a general heuristic, whenever you feel a pang of jealousy, that's an emotional cue that
you ought to be deconstructing whatever it is that that person is doing. 

tk flesh out

\section{Names and keywords}

\begin{quote}
  If you have the name of a spirit, you have power over it.
  \attrib{Hal Abelson}
\end{quote}

\newthought{Okay, now on} to more traditional forms of research: papers, books,
data, that sort of jazz.

The most important piece of research, second only to realizing that you have a
problem, is figuring out what things are called and what language to do.

This is a daunting task. How do I find the right keywords without already
knowing the keywords? You have a few options.

\section{Human resources and expertise}

The fastest and most effective way to figure out what something is called, or
where to look, is to take advantage of human expertise.

Here's a real example I encountered the other day: someone on the internet was
complaining that they couldn't find any articles that not only dealt with energy
costs, but also included the cost to the environment.

Now, depending on your facility with economics, you'll realize that what this
guy is looking for is a paper on energy cost that includes ``externalities,'' a
piece of economic jargon that refers to the ``cost or benefit that affects a
party who did not choose to incur that cost or benefit.''

In this case, the fastest way for this guy to find what he was looking for would
be to talk to someone more familiar with economic jargon, maybe by emailing a
professor or asking a friend with an interest in the subject.

The second option is actually representative of a broader method: leveraging the
collective intelligence and expertise of your social network. If you have a lot
of Facebook friends, Twitter follows, a popular blog, etc., you can leverage
this. Throw out a question to them.

Most of the time, people jump at a chance to look knowledgeable.\footnote{After
  all, what do you think the point of the externality example was?}

\section{Q\&A Sites}

If you don't directly know someone with the relevant expertise, you can harness
the brainpower of a good chunk of the internet by asking your question on a
question and answer site, like one of the Stack Exchanges or on Quora.

If you describe your problem succintly, you'll often receive a prompt and useful
response from, you know, a flesh and blood human who can point you in the right
direction.

- tk, link to best textbook on every subject

\section{Librarians}

If you can't find someone with domain expertise to help, your best bet is to get
in touch with a librarian.

They're paid to help people find what they're looking for, and they're pretty
good at it. They'll be familiar with the available information databases, how to
interpret catalogue numbers, and all of that. 

\section{Recommendation engines}

If you have the name of a relevant resource, like a book, you can pull it up in
Amazon and browse through the related items. You'll often find closely related
material to dig into.

For a paper, if you pull it up in Google Scholar, and click ``related
articles,'' which will give you, you guessed it, related material.

\newthought{Further, if you} bother tracking down a book in the library, check out whatever
is next to it and on the same row. Chances are, you'll pull up a lot more stuff
that will be useful.

Indeed, if you have the name of a book, you can track down its category in the
library of congress and, once you know the name of a subject, you can use that
to track down more books.

\newthought{This is} a subset of a broader strategy. As long as you have one
book or paper, you can use that to find even more material. It's a bit like
there's the top of a root sticking out of the ground and, once you get a hold of
it, soon you're yanking up a few miles of the stuff plus a tree.

The most straightforward and effective way to go about this is to track down all
of the sources mentioned in the text that you've discovered. If you have one
relevant paper, it'll typically review related research either in the
introdcution or at the very end of the paper. You can then track those down, and
then do thing recursively, until you've amassed about everything on your
subject.

This same method applies to books and, with Google Scholar, you can also go in
the other direction: hunt down everything that cites a certain paper of book.

\section{Bibliographies}

An annotated bibliography of a subject is just what it sounds like: a list of
works along with a description of each work. Often, these bibliographies will
also indicate the most important papers in a field.

These are worth their weight in gold and, once you have the
name of your subject, Google searching for one should be your first
priority. This can be accomplished with a search like, ``jugment and decision
making annotated bibliography.''

If you don't have the name of your subject, but you do have the name of a
relevant book or paper, you can sometimes find a useful resource with a search
for the name of the resource, along with the words bibliography of reading list,
e.g. ``"Thinking and Deciding" reading list'' or ``"Thinking and Deciding"
bibliography''.

\section{Textbooks}

I wanted to mention review articles, but first I should cover the grandaddy of
them all: textbooks. Textbooks are like extended review articles.

Introductory textbooks are designed for people new to a field, to get them up to
speed as quickly as possible. They're made for teaching. Other resources, like
papers, are often aimed at audiences already familiar with the material.

But I should add a caveat: if a textbook isn't good, it's often terrible. And
they often suffer from feature bloat, so it'll end up being like a 1200 page
behemoth that tries to be everything, when all you really wanted was a concise
introduction.

If you're faced with this problem, you can often find lecture notes published
online. Since these are constrained by classroom time in a way that textbooks
aren't, they can be more useful for getting an overview of a field.

When looking for a good textbook, you're tasked with more research: check out
Amazon reviews and reviews on Goodreads and what textbook college courses on the
subject use. A search for ``best textbook introduction to x'' is often
effective.

\section{Review articles}

As I mentioned earlier, textbooks are like extended review articles. The
non-extended form, well, they're just review articles.

To find them, search Google Scholar for ``subject review article.'' In general,
Google Scholar is a good place to begin looking for published research on just
about anything. Just plug in a keyword and go.

Oh, and another point: \textbf{research papers are almost always easier to
  understand than you initially expect.} Excepting highly technical fields like
pure mathematics (and I'm not even sure about that), non-experts can often get a
solid understanding of the gist of a paper, even if they can't follow all of the
details, which are usually not that important anyways.

At this point, I've read somewhere between 150 and 300 papers, and I can think
of only one that I wanted to read but couldn't wrap my head around, and that was
a rather esoteric mathematical piece.

\section{OEIS}

If you're interested in mathematics, the ``On-Line Encyclopedia of Integer
Sequences'' has been quietly revolutionizing the way to look up information
about some mathematical phenomenon.

If you can get an integer sequence out of whatever it is that you're studying,
you can look it up and figure out what's been written about it.

Note that human resources are especially valuable when it comes to mathematics,
as our current search technologies don't handle mathematical discovery very
well.

For example, you can't exactly just Google an equation like equation 1 here.\foonote{it's one of the Navier-Stokes equations.}

\begin{align}
      \pdv{(\rho e)}{t}+\vec{\nabla}\cdot(\rho e+p)\vec{u} &= \vec{\nabla}\cdot(\Bar{\Bar{\tau}}\cdot\vec{u})+\rho\vec{f}\vec{u}+\vec{\nabla}\cdot\vec{\dot{q}}+r
\end{align}

So, for integer sequences, try the OEIS. For other math, try asking a friend, or
one of the Q\&A sites, like the Math StackExchange (tk add link).

\section{Summary: How to find information}

Alright, so I've reviewed here how to find information, but it's been a sort of
lightning tour. If all this seems overwhelming, just follow these steps:

\begin{enumerate}
\item Figure out your topics keywords, what it's called
  \begin{itemize}
  \item Take advantage of human resources
    \begin{itemize}
    \item Ask a friend or colleague
    \item Email someone in the field\footnote{Worst case scenario: you get back a
      passive-agressive email signed with a professor's full title.}
    \item Ask on a Q\&A site
    \item Ask a librarian. Seriously, that's what they're there for!
    \end{itemize}
  \item If you already know of relevant resources, you can use those to discover
    more. (See below.)
  \end{itemize}
\item Once you have your keyword, amass everything vaguely related to it:
  \begin{itemize}
    \begin{itemize}
    \item Search for a textbook on the topic.
    \item Search for a review article on the textbook.
    \item Search for an annotated bibliography.
    \item Check the references on Wikipedia for your subject.
    \end{itemize}
  \item Once you have a book on the subject, try:
    \begin{itemize}
    \item Search for the book on Amazon and search for related works on
      ``Customers Who Bought This Item Also Bought.''
    \item Search for the book on Google Scholar, click ``Related articles'', and
      scan through those. Also check out who cites that book.
    \item Check for a bibliography at the end of the book and hunt down those
      papers.
    \item Search Google for ``reading list "bookname"'', ``annotated
      bibliography "bookname"', or ``recommended reading "bookname"'
    \item Once finished, repeat this process for everything new you found.
    \end{itemize}
  \item The process is similar for papers:
    \begin{itemize}
    \item Search for the paper on Google Scholar, hunt down ``Related articles''
    \item Check out everything the paper cites and everything cited by it.
    \item Search Google for ``reading list "paper title"'' or ``annotated
      bibliography "paper title"', or ``recommended reading "bookname"'
    \item Repeat this for every item you just discovered.
    \end{itemize}
  \end{itemize}
\end{enumerate}

\section{Filtering}


Stuff to include:

- A few tips on effective googling, keywords.
- A note on the value of information. You can sell it. (Maybe an intro to VoI caluclations, too?)
- On the effectiveness of filtering. 
- Skimming, table of contents.
- Reading papers is remarkably easy.
- Further reading.
