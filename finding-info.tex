\chapter{Finding, Filtering, and Managing Information}
\begin{quote}
  In an information-rich world, the wealth of information means a dearth of something else: a scarcity of whatever it is that information consumes. What information consumes is rather obvious: it consumes the attention of its recipients. Hence a wealth of information creates a poverty of attention and a need to allocate that attention efficiently among the overabundance of information sources that might consume it.
\attrib{Herbert Simon \cite{simon1971designing}}
\end{quote}

\newthought{In a classroom evironment,} at almost every level, students are given the sources that they're expected to gather information from. They're assigned a textbook or papers to read. Students are never in a position where they're forced to evaluate the quality of several textbooks before deciding which to sink time into.

How to do research, then, is a skill that's rarely taught, and almost never taught well and, of course, that's where this text comes in. There are, broadly speaking, four steps to information gathering.

\begin{enumerate}
  \item Realizing that information would improve of illuminate the situation
  \item Gathering all possibly relevant information and sources
  \item Filtering out the irrelevant bits
  \item Processing the information and recording what's relevant
\end{enumerate}

\section{Recognizing an opportunity}

\newthought{Today, the world is awash} in information, a constant deluge of the stuff. Supposing that tomorrow you were transported to some platonic library, where you never needed to sleep or eat, such that you could read constantly, it would still take you about 62,000 years to read all the books that currently exist. \marginnote{``Basically what Shannon said is that we define information as the opposite of uncertainty. If you have uncertainty, you don't have information. And once you receive information, uncertainty is being resolved.'' Martin Hilbert}

If that is not tantalizing enough, consider the case of Shaun Winterton. One May day in 2011, he discovered a new species of green lacewing, a dragonfly-esque insect.

Where did he find it, you might wonder -- hiking in the Amazon, maybe? But no. He discovered it from the comfort of his own home while browsing the photo-sharing website Flickr. A new species! Advancing human knowledge! From his couch.\cite{winterton2012charismatic}\footnote{The story continues. Winterton contacted the photographer, Guek Hock Ping, and asked him if he had a specimen. He didn't, but eventually caught one and sent it to Winterton. Winterton then named the species after his own daughter, not Guek.} 

So, yeah, there is a lot of information out there. If, for some real-life problem, you can't think of any already existing information to take advantage of, you're not thinking hard enough.

I've written, for instance, a popular article on the science of eye contact. It opens with an anecdote about 17th century Italian women: in an attempt to enhance their appearance, they would use the essence of the Belladonna plant to dilate their pupils. The only problem? Belladonna, sometimes called nightshade, is poisonous.

Now, I think this is a great fact, the sort of spice that sets an article apart from more common fare. How'd I find it? I did a Google search for ``site:scientificamerican.com eye contact''. The addition of the ``site:'' operator restricts my results to article from that site. I then read the four or five relevant articles that Google returned. One of them included that tidbit -- I don't know where they got it. Perhaps doing the same thing.

Another example? You know that bit about Shaun Winterton discovering the green lacewing on Flickr? Yeah, I found that with a Google search for ``site:scientificamerican.com finding information.''\footnote{Why \textit{Scientific American}? No particular reason beyond the wonderful tendency of science writers to pack their articles with just the sort of anecdote I'm after.}

\newthought{When people think} of research, a few representative examples spring to mind: a man in a white coat, a laboratory, a library, heavy books, that sort of thing.

It's so much broader than that! If you're looking for a long-term relationship and, as part of that process, you take someone out on a date: \textit{that's research}. When you ask a friend how he feels about his new job, more research! Deconstructing your competitor's last successful ad campaign? That's research.\footnote{Writing this book came about as the result of research. I noticed that so many successful blogs were running email campaigns and offering an incentive for signing up and, well, now we're here.}

Research is trying on clothes until you find something that fits your shoulders, instead of draping across them like a dead raccoon. Research is paying attention to the words that your customers use in their emails so that, next time you need content for your company blog, you aren't stuck wondering, ``What do I title this?'' Research is trying new foods, 

More examples of research:


Stuff to include:

- Textbooks are the best way to get up to date on a subject.
- OEIS
- Human expertise is still immensely beneficial in today's world.
- Annotated bibliographies are worth their weight in gold.
- the important of knowing the name of a keyword, example: externalities. quote: know the name of a spirit, have control over it.
- A few tips on effective googling, keywords.
- Google scholar.
- A note on the value of information. You can sell it. (Maybe an intro to VoI caluclations, too?)
- On the effectiveness of filtering. 
- Skimming, table of contents.
- Reading papers is remarkably easy.
- Serendipity: looking at library racks

On August 7th, Zookeys published a paper on the discovery of the Semachrysa jade, a new species of the insect green lacewing. The discovery was noteworthy enough to be picked up by Science two days later because Shaun Winterton, the primary researcher, didn’t encounter the insect in its native Malaysia, but on the photo-sharing website Flickr.
