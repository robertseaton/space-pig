\chapter{Finding, Filtering, and Managing Information}
\begin{quote}
  In an information-rich world, the wealth of information means a dearth of something else: a scarcity of whatever it is that information consumes. What information consumes is rather obvious: it consumes the attention of its recipients. Hence a wealth of information creates a poverty of attention and a need to allocate that attention efficiently among the overabundance of information sources that might consume it.
\attrib{Herbert Simon\cite{simon1971designing}}
\end{quote}

Stuff to include:

- Textbooks are the best way to get up to date on a subject.
- OEIS
- Human expertise is still immensely beneficial in today's world.
- Annotated bibliographies are worth their weight in gold.
- the important of knowing the name of a keyword, example: externalities. quote: know the name of a spirit, have control over it.
- A few tips on effective googling, keywords.
- Google scholar.
- A note on the value of information. You can sell it. (Maybe an intro to VoI caluclations, too?)
